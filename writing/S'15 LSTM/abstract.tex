\begin{abstract}
Tree-structured neural networks aim to deliver a robust and principled method for representing sentence meaning, but these models largely have not outperformed simpler sequence-based models by substantial margins. We hypothesize that sequence models like LSTMs are able to discover and implicitly use the same kinds of recursive compositional structures that the tree-structured are built around, mitigating the advantage of the tree-structured models. We investigate this possibility by evaluating both models on an artificial talk for which recursive compositional structure is crucial, and find that the sequence model is able to identify and exploit the underlying structure, though that it is less efficient at learning than the tree-structured models, only succeeding after exposure to a larger and richer set of training data.
\end{abstract}
