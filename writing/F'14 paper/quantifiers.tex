\section{Reasoning with quantifiers and negation}\label{sec:quantifiers}

We have seen that recursive models can learn an approximation of propositional
logic.  However, natural languages can express functional meanings of
considerably greater complexity than this.  As a step towards
investigating whether our models can capture this complexity, we now
directly measure the degree to which they are able to
develop suitable representations for the semantics of natural language
quantifiers like \ii{most} and \ii{all} as they interact with negation. Quantification 
and negation are far from
the only place in natural language where complex functional meanings
are found, but they are natural starting point, since they can be tested
in sentences whose structures are otherwise quite simple, and since they have
formed a standard case study in prior formal work on natural
language inference \cite{Icard:Moss:2013:LILT}.

\paragraph{Experiments}
Our experimental data consist of pairs of sentences generated from a
small artificial grammar. Each sentence contains a quantifier, a noun
which may be negated, and an intransitive verb which may be
negated. We use the basic quantifiers \ii{some}, \ii{most}, \ii{all},
\ii{two}, and \ii{three}, and their negations \ii{no}, \ii{not-all},
\ii{not-most}, \ii{less-than-two}, and \ii{less-than-three}. We also
include five nouns, four intransitive verbs, and the negation symbol
\ii{not}. In order to be able to define relations between sentences
with differing lexical items, we define the lexical relations between
each noun--noun pair, each verb--verb pair, and each
quantifier--quantifier pair. The grammar accepts aligned pairs of
sentences of this form and calculates the natural logic relationship
between them.  Some examples of these data are provided in (\ref{p1}--\ref{p3}) below.
  As in previous sections, the goal of
learning is then to assign these relational labels accurately to
unseen pairs of sentences.

%nouns = ['warthogs', 'turtles', 'mammals', 'reptiles', 'pets']
%verbs = ['walk', 'move', 'swim', 'growl']
%dets = ['all', 'not_all', 'some', 'no', 'most', 'not_most', 'two', 'lt_two', 'three', 'lt_three']
%adverbs = ['', 'not']

% To assign relation labels to sentence pairs, we built a small
% task-specific implemenation of MacCartney's logic that can
% accurately label sentences of this restricted language. The logic is
% not able to derive all intuitively true relations of this language,
% and fails to derive a single unique relation for certain types of
% statement, including De Morgean's laws (e.g. \ii{(all pets) growl
% $\natneg$ (some pet) (not growl)}), and we simply discard these
% examples. Exhaustively generating the valid sentences under this
% grammar and choosing those to which a relation label can be assigned
% yields 66k sentence pairs. Some examples of these data are provided
% in Table~\ref{examplesofdata}.

In each run, we sample 20\% of the 60k generated pairs to test on and 
train on the remainder. In this setting, each model must learn a
complete reasoning system for the limited language and logic presented
in the training data, and must be able
to recognize all of the lexical relations between the nouns, verbs,
and quantifiers and how they interact. For instance, it might see
pairs like \eqref{p1} and \eqref{p2} in training and be required to 
then label \eqref{p3}.

\vspace{-0.6cm}
\begin{gather}
  \text{(most turtle) swim} \natalt \text{(no turtle) move}\label{p1}
  \\
  \text{(all lizard) reptile} \natfor  \text{(some lizard) animal}\label{p2}
  \\
  \text{(most turtle) reptile} \natalt \text{(no turtle) animal}\label{p3}
\end{gather}\vspace{-0.8cm}

\begin{table}[tp]
  \centering \small
  \begin{tabular}{ l r@{ \ }r r@{ \ }r r@{ \ }r }
    \toprule
    ~&\multicolumn{2}{c}{$\natind$ only} & \multicolumn{2}{c}{25d RNN}  & \multicolumn{2}{c}{25d RNTN}\\
    \midrule
    Train & 35.4 &(7.5) &	\textbf{100}&		\textbf{(100)}&	99.5&	(99.3)\\	
    Test & 35.4 &(7.5) &	\textbf{99.4}&\textbf{(99.5)}& \textbf{99.4} & (99.2)\\
    \bottomrule
  \end{tabular}
  \caption{Performance on the quantifier experiments, given as \% correct and macroaveraged F1.}
  \label{qresultstable}
\end{table} 
%
% do not allow a blank line --- adds too much space
%
\paragraph{Results} The results, reported in Table~\ref{qresultstable}, show that both models are able to learn to generalize the underlying logic satisfactorily. We did not observe any consistent pattern in the few errors made by either model, and errors did not tend to be consistent across runs of the experiment, suggesting that there is no fundamental obstacle to learning a perfect model for this problem.

% TODO: Baseline