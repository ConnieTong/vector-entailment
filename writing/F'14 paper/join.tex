
\section{Reasoning about semantic relations}\label{sec:join}

 % TODO: Why is it okay to study non-recursive models here.
 % TODO: Run a baseline for shits and giggles?
 % TODO: Mention that we do an easier version of this implicitly elsewhere.

If a model is to learn the behavior of a relational logic like ours
from a finite amount of data, it must learn to deduce new
relations from seen relations in a sound manner. The simplest such
deductions involve atomic statements using the relations in
Table~\ref{b-table}. 
For instance, from the relation $p_1 \natrev p_2$ between two propositions, 
one can conclude that $p_2 \natfor p_1$ using the definitions of the relations directly. 
If one is also given the relation $p_1 \natrev p_3$, one can conclude that $p_1 \natrev p_2$, by basic
set-theoretic reasoning (transitivity of $\natrev$). Similarly, from
$p_1 \natfor p_2$ and $p_2 \natneg p_3$, it follows that $p_1 \natalt p_3$.  The
full set of sound such inferences on pairs of premise relations is depicted in
Table~\ref{tab:jointable}.

% about the relations themselves that do not depend on the
% internal structure of the things being compared. For example, given
% that $a\sqsupset b$ and $b\sqsupset c$ one can conclude that
% $a\sqsupset c$ by the transitivity of $\sqsupset$, even without
% understanding $a$, $b$, or $c$. These seven relations support more
% than just transitivity: MacCartney and Manning's
% \cite{maccartney2009extended} join table defines 32 valid inferences
% that can be made on the basis of pairs of relations of the form $a R
% b$ and $b R' c$, including several less intuitive ones such as that if
% $a \natneg b$ and $b~|~c$ then $a \sqsupset c$.

\begin{table}[htp]
  \centering  \small
  \setlength{\arraycolsep}{8pt}
  \renewcommand{\arraystretch}{1.1}
  \newcommand{\UNK}{\cdot}  
  $\begin{array}[t]{c@{ \ }|*{7}{c}|}
    %\hline
    \multicolumn{1}{c}{}
             & \nateq     & \natfor     & \natrev     & \natneg    & \natalt     & \natcov     & \multicolumn{1}{c}{\natind} \\
    \cline{2-8}
    \nateq  & \nateq &   \natfor &  \natrev &  \natneg &   \natalt &  \natcov &  \natind \\
    \natfor & \natfor &  \natfor &  \UNK &  \natalt &   \natalt &  \UNK &  \UNK \\
    \natrev & \natrev &  \UNK &  \natrev &  \natcov &   \UNK &  \natcov &  \UNK \\
    \natneg & \natneg &  \natcov &  \natalt &  \nateq &    \natrev &  \natfor &  \natind \\
    \natalt & \natalt &  \UNK &  \natalt &  \natfor &   \UNK &  \natfor &  \UNK \\
    \natcov & \natcov &  \natcov &  \UNK &  \natrev &   \natrev &  \UNK &  \UNK \\
    \natind & \natind & \UNK &  \UNK &  \natind &  \UNK &  \UNK &  \UNK \\
    \cline{2-8}
  \end{array}$
  \caption{Inference path from premises $p_1Rp_2$ (row) and $p_2Sp_3$ (column) to the relation that holds between $p_1$ and $p_3$, if any.  These inferences are based on basic set-theoretic truths about the meanings of the underlying relations as described in Table~\ref{b-table}. We assess our models' ability to reproduce such inferential paths. Cells containing a dot correspond to pairs
of relations for which no valid inference can be drawn.} % TODO: Move harder example in to replace caption.
  \label{tab:jointable}
\end{table}

\paragraph{Experiments}
To generate data for these inferential patterns, we
create small boolean structures for our logic in which terms denote
sets of entities from a small domain.  Figure~\ref{lattice-figure}
depicts a structure of this form. The lattice gives the full model,
and we generate a relational statement for each pair of labels in the model. 
We divide these statements evenly into training and test sets, and remove from the
test set statements which cannot be proven from the training
statements.
In each experimental run, we create 80 randomly generated sets drawing from
a domain of 7 elements, yielding a training set of 3200 relations and a test set of 
2960 relations.
% TODO: Explain what the lattice shows.

We trained models with both the NN and NTN comparison functions on these
data sets.\footnote{Since this task relies crucially on the learning of a pair of vectors, no strictly simpler model is likely viable as a baseline.} In both cases, the models are implemented as
described in Section~\ref{methods}, but since the items being compared
are single terms rather than full tree structures, the composition
layer is not used, and the two models are not recursive. We simply present
the models with the (randomly initialized) embedding vectors for each
of two terms, ensuring that the model has no information about the terms
being compared except for the relations between them that appear in training.


\begin{figure}[t]
  \centering
  \begin{subfigure}[t]{0.45\textwidth}
    \centering
    \newcommand{\labelednode}[4]{\put(#1,#2){\oval(1.5,1)}\put(#1,#2){\makebox(0,0){$\begin{array}{c}#3\\\{#4\}\end{array}$}}}
    \setlength{\unitlength}{1cm}\scalebox{0.8}{
    \begin{picture}(5,5.5)
      \labelednode{2.50}{5}{}{a,b,c}
      
      \put(0.75,4){\line(3,1){1.5}}
      \put(2.5,4){\line(0,1){0.5}}
      \put(4.25,4){\line(-3,1){1.5}}
      
      \labelednode{0.75}{3.5}{p_1,p_2}{a,b}
      \labelednode{2.50}{3.5}{p_3}{a,c}
      \labelednode{4.25}{3.5}{p_4}{b,c}
      
      \put(0.75,2.5){\line(0,1){0.5}}
      \put(0.75,2.5){\line(3,1){1.5}}
      
      \put(2.5,2.5){\line(-3,1){1.5}}
      \put(2.5,2.5){\line(3,1){1.5}}
      
      \put(4.25,2.5){\line(0,1){0.5}}
      \put(4.25,2.5){\line(-3,1){1.5}}
      

      \labelednode{0.75}{2}{p_5,p_6}{a}
      \labelednode{2.50}{2}{}{b}
      \labelednode{4.25}{2}{p_7,p_8}{c}
      
      \put(2.5,1){\line(-3,1){1.5}}
      \put(2.5,1){\line(0,1){0.5}}
      \put(2.5,1){\line(3,1){1.5}}
      
      \labelednode{2.5}{0.5}{}{}
    \end{picture}}
    \caption{Simple boolean structure. The letters name the sets. Not all sets have names, and
    some sets have multiple names, so that learning $\nateq$ is non-trivial.}
  \end{subfigure}
  \qquad
    \begin{subfigure}[t]{0.43\textwidth}
    \centering \vspace{0.4cm}
    \setlength{\tabcolsep}{12pt}
    \begin{tabular}[b]{c  c}
      \toprule
      Train & Test \\
      \midrule
      $p_1 \nateq p_2$              & $p_2 \natneg p_7$ \\
      $p_1 \natrev p_5$              & $p_2 \natrev p_5$ \\
      $p_4 \natrev p_8$              & \strikeout{$p_5 \nateq p_6$} \\
      $p_5 \natalt p_7$              & \strikeout{$p_7 \natfor p_4$} \\
      $p_7 \natneg p_1$           & $p_8 \natfor p_4$ \\

      \bottomrule
    \end{tabular}

    \caption{A few examples of atomic statements about the
      model.  Test statements that are not provable from the training data shown are
      crossed out.}
  \end{subfigure}  
  \caption{Small example structure and data for learning relation composition.}
  \label{lattice-figure}
\end{figure} 

\begin{table}[tp]
  \centering \small
  \begin{tabular}{ l r@{ \ }r r@{ \ }r r@{ \ }r }
    \toprule
    ~&\multicolumn{2}{c}{$\natind$ only} & \multicolumn{2}{c}{15d RNN}  & \multicolumn{2}{c}{15d RNTN}\\
    \midrule
    Train &53.8 &(10.5)    & 99.8&	(99.0) & \textbf{100} & \textbf{(100)}\\
    Test &52.0 &(10.3) &	94.0&(87.0)& \textbf{99.6} & \textbf{(95.5)}\\
    \bottomrule
  \end{tabular}
  \caption{Performance on the semantic relation experiments. These results and all other results on artificial data are reported as mean accuracy scores over five runs followed by mean macroaveraged F1 scores in parentheses.}
  \label{joinresultstable}
\end{table}

\paragraph{Results} 
Table \ref{joinresultstable} shows the results. 
The NTN is able to accurately encode the relations 
between the terms in the geometric relations between their vectors, 
and is able to then use that information to recover relations that 
are not overtly included in the training data. The NN also generalizes fairly well, 
but makes enough errors that it remains an open question whether 
it is capable of learning representations with these properties. 
We cannot rule out the possibility that different optimization techniques or
further hyperparameter tuning could lead an NN model to succeed here.

In one run for example, both models successfully classified the test example $p_1 \natfor p_3$, potentially learning from the training examples $\{p_1 \natfor p_{51},~p_3 \natrev p_{51}\}$ or $\{p_1\natfor p_{65},~p_3 \natrev p_{65} \}$. On a somewhat similar example involving comparably frequent relations, the RNTN correctly labeled $p_6 \natrev p_{24}$, likely on the basis of the training examples $\{p_6 \natcov p_{28},~p_{28} \natneg p_{24}\}$, while the RNN incorrectly assigned it $\natind$.

% From train\test_1