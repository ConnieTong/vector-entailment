% \begin{abstract}
% Recursive neural network models for sentence meaning have been successful in many tasks, but it remains an open question whether any such model can learn compositional semantic representations that support logical deduction. We pursue this question directly by evaluating whether two such models can correctly learn to identify logical labelships such as entailment and contradiction between sentences. We first describe a recursive neural tensor network model for entailment and demonstrate it on the SICK challenge. We then use artificial data from a logical grammar to test the model's ability to learn to handle basic labelal reasoning, recursive functions, and quantification. The model performs competitively on the SICK data and generalizes well in all three experiments on simulated data, suggesting that it can learn suitable representations for logical inference in natural language.
% \end{abstract}


%%%% POTTS REVISION
\begin{abstract} 
  Recursive neural networks (RNNs) for sentence meaning have been
  successful for many tasks, but it remains an open question whether
  they can learn compositional semantic representations that
  support logical deduction. We pursue this question by
  evaluating whether two such models---plain RNNs and recursive neural
  tensor networks (RNTNs)---can correctly learn to identify logical
  labelships such as entailment and contradiction.
  In our first set of experiments, we generate artificial data from a
  logical grammar and use it to evaluate the models' ability to learn
  to handle basic labelal reasoning, recursive structures, and
  quantification. We then evaluate the models on the more natural
  SICK challenge data. Both models perform competitively on the SICK
  data and generalize well in all three experiments on simulated data,
  suggesting that they can learn suitable representations for logical
  inference in natural language. 
\end{abstract}
