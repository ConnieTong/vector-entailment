\section{Conclusion}\label{sec:conclusion}

Natural languages are powerful vehicles for complex, flexible
commonsense reasoning, and nearly all questions about meaningfulness
in language can be reduced to questions of entailment
and contradiction in context. This suggests that NLI is an ideal testing ground
for theories of semantic representation, and that training for NLI
tasks can provide rich domain-general semantic representations.  To
date, however, it has not been possible to fully realize this
potential due to the limited nature of existing NLI resources.  This
paper sought to remedy this with a new, large-scale, naturalistic
corpus of sentence pairs labeled for entailment, contradiction, and
independence. We used this corpus to evaluate a range of models,
and we found that simple lexicalized models and neural network
models perform competitively, and that the representations learned by
a neural network model from our corpus can be used in transfer learning
to dramatically improve performance on a standard challenge dataset.
This result serves as powerful new evidence that neural representations can, 
when given a sufficiently extensive and challenging task, 
be trained to accurately encode rich, multifaceted semantic information, and it invites
further research into the problem of learning maximally faithful and expressive
 representations.