\section{SNLI as a platform for NLI research}
\todo{write me}

%
% Baselines
%
\subsection{Baselines}
We evaluate a number of strong baselines on the SNLI corpus.
The first class of baselines use models from the Excitement Open
  Platform (EOP) 
  \cite{pado2014design,magnini2014excitement}
  -- an open source platform for RTE research which
  is distributed alongside a number of RTE pipelines.
We additionally evaluate against a both an unlexicalized and lexicalized
  classifier with straightforward feature sets.
\todo{some meaningful insight from \#s}


%
% EOP
%
\paragraph{Excitement Open Platform}
% what is EOP
The Excitement Open Platform is a tool to quickly develop RTE systems
  while sharing components such as common lexical resources and 
  evaluation sets.
% 9 systems compared against
A number of systems have been built using the platform, 9 of them
  applicable to English are publicly distributed with version 1.2.1
  of the software.
% these fall into 2 classes
These fall into two classes of algorithms: 2 are edit distance based,
  whereas the remaining 7 make use of different features in a
  maximum entropy classifier.

% methodology
We convert the 3-way classification task in SNLI into the RTE setting
  by labeling both the \unknown\ and \contradiction\
  labels as negative entailment, and treating the \entailment\ label as
  the positive entailment.
This creates a biased dataset of 66\% negative examples.
% we report the best results from each class
We run each of the 9 algorithms distributed with EOP on the 2-class
  SNLI dataset, and report results for the best edit distance 
  configuration and the best classifier based configuration, as
  determined by performance on the development set.
\todo{we may not have CPU time for this}.

%
% EOP RESULTS TABLE
%
% Some definitions
\def\t#1{\small{#1}}
\def\b#1{\t{\textbf{#1}}}
\def\colspaceS{2.0mm}
\def\colspaceM{3.0mm}
\def\colspaceL{4.0mm}

% The table
\begin{table}
\begin{center}
\begin{tabular}{l@{\hskip \colspaceL}c@{\hskip \colspaceS}c@{\hskip \colspaceS}c@{\hskip \colspaceS}c@{\hskip \colspaceL}c@{\hskip \colspaceS}c}
\hline
\textbf{System} & \multicolumn{4}{c}{\b{SNLI}} & \multicolumn{2}{c}{\b{RTE-3}} \\
 & \t{P} & \t{R} & \t{F$_1$} & \t{Acc.} & \t{F$_1$} & \t{Acc.} \\
\hline
\t{Edit Distance} & \t{??.?} & \t{??.?} & \t{??.?} & \t{??.?} & 
                    \t{??.?} & \t{??.?} \\
\t{Classifier}    & \t{68.0} & \t{38.7} & \t{49.3} & \t{73.5} & 
                    \t{64.2} & \t{65.3} \\
\hline
\end{tabular}
\end{center}
% The caption
\caption{
\label{tab:eopresults}
2-class accuracy and precision/recall of existing competitive RTE
systems from the Excitement Open Platform.
Results are reported for an edit-distance based algorithm, and a
  classifier-based system.
For comparision, results are also reported on the RTE-3 dataset.
\todo{numbers are approx.}
}
\end{table}
%
% END EOP RESULTS TABLE
%

% the best systems
We report results in \reftab{eopresults}.
The best edit distance algorithm tunes the the weight of the three 
  case-insensitive edit distance operations on the training set, 
  after removing stop words.
The best classifier-based system \todo{does something}.
\todo{some insight from results}.

%
% Lexicalized Classifier
%
\paragraph{Lexicalized Classifier}
Unlike the RTE datasets, SNLI's size allows approaches which make use of
  rich lexicalized features.
We implmement a lexicalized classifier to provide a strong baseline for future work
  on the dataset.
Our classifier implements 6 features:
\begin{enumerate}  % TODO(gabor) This takes up way too much space
\setlength\itemsep{0em}
  \item the BLEU score of the \hypothesis\ with respect
  to the \premise, using an n-gram length between 1 and 4.

  \item The length difference between the \hypothesis\ and the \premise, as a real-valued
  feature.

  \item The overlap between words in the \premise\ and \hypothesis. This is counted
  both as an absolute count and a percentage of possible overlap, and is computed
  both over all words and over just nouns, verbs, adjectives, and adverbs.

  \item For every pair of words across the \premise\ and \hypothesis\ which share a 
  POS tag, an indicator feature over the two words.
  
  \item For every pair of bigrams across the \premise\ and \hypothesis\ which share a 
  POS tag, an indicator feature over the two bigrams.
  
  \item An indicator for every unigram and bigram in the \hypothesis.
\end{enumerate}

%
% BOW RESULTS TABLE
%
% Some definitions
\def\t#1{\small{#1}}
\def\b#1{\t{\textbf{#1}}}
\def\colspaceS{2.0mm}
\def\colspaceM{3.0mm}
\def\colspaceL{4.0mm}

% The table
\begin{table}
\begin{center}
\begin{tabular}{l@{\hskip \colspaceL}c@{\hskip \colspaceS}c@{\hskip \colspaceS}c@{\hskip \colspaceS}c@{\hskip \colspaceL}c}
\hline
\textbf{System} & \multicolumn{4}{c}{\b{2-Class}} & \b{3-Class} \\
 & \t{P} & \t{R} & \t{F$_1$} & \t{Acc.} & \t{Acc.} \\
\hline
\t{Lexicalized}            & \t{??.?} & \t{??.?} & \t{??.?} & \t{??.?} & \t{77.3} \\
\t{Unigrams Only}          & \t{??.?} & \t{??.?} & \t{??.?} & \t{??.?} & \t{??.?} \\
\t{Unlexicalized}          & \t{??.?} & \t{??.?} & \t{??.?} & \t{??.?} & \t{??.?} \\
\hline
\end{tabular}
\end{center}
% The caption
\caption{
\label{tab:bowresults}
Accuracy on the 2-class and 3-class SNLI dataset, as well as precision/recall
  on the 2-class task to allow for comparision with the RTE systems.
Ablation studies are reported for removing bigram features and all lexical
  features.
\todo{numbers are approx.}
}
\end{table}
%
% END BOW RESULTS TABLE
%

\Fig{learning_curves_bow.pdf}{0.6}{bowlearncurve}{
A learning curve for the lexicalized and unlexicalized baseline classifiers.
Note the y axis starting at a random-chance accuracy of 33\%.
\todo{numbers are approx.}
}

% Results
We report results in \reftab{bowresults}, along with ablation studies for removing
  the bigram features (leaving only the unigram entailment pair feature),
  and for removing all lexicalized features.
In addition, we report learning curves for the lexicalized and fully unlexicalized
  classifiers in \reffig{bowlearncurve}.

Note that \todo{something interesting happens}.


%\subsection{Existing NLI systems}

\subsection{Sentence embeddings and NLI}

\subsection{Results}
