\section{Introduction}\label{sec:introduction}

The semantic concepts of entailment and contradiction are central to
all aspects of natural language meaning
\cite{Katz72,vanBenthem08NATLOG}, from the lexicon to the content of
entire texts. Thus, \emph{natural language
  inference} (NLI) --- characterizing and using these relations in
computational systems
\cite{dagan2006pascal,MacCartney09,maccartney2009extended} --- is
essential in tasks ranging from information retrieval to semantic
parsing to commonsense reasoning.

NLI has been addressed using a wide variety of techniques, including
those grounded in syntactic structures, knowledge bases, and symbolic
logic. In recent years, it has become an important testing ground for
approaches employing \emph{distributed} word and phrase
representations. Distributed representations excel at capturing
relations based in similarity, and they have proven effective at
modeling simple dimensions of meaning like evaluative sentiment
\cite{socher2013acl1}, but it is less clear that they can be
trained to support the full range of logical inferences required for
NLI. In the SemEval 2014 task aimed at evaluating distributed
representations for NLI, the best-performing systems relied heavily on
additional features and reasoning capabilities
\cite{marelli2014semeval}.

Our primary objective in this paper is to provide a new empirical
evaluation of a wide range of models for distributed semantic
representations in the context of NLI. However, in our view, the
existing corpus resources in this area do not permit such an
assessment. They are generally too small for training modern
data-intensive, wide-coverage models, and they are often beset with
indeterminacies of event and entity coreference that significantly
impact annotation quality.

To address this, we present a new corpus of sentence pairs labeled for
entailment, contradiction, and semantic independence. At 550,152
sentence-pairs, this corpus is orders of magnitude larger than all
other resources of its type. And, in contrast to many such resources,
all of the sentences were written by humans in a grounded,
naturalistic context. In a separate validation phase, we collected
four additional judgments for each label for 56,941 of the examples,
and 98\% of cases emerge with a gold label, which attests to the high
quality of the data.

In this paper, we use this corpus to evaluate a wide variety of models
for natural language inference, including rule-based systems, simple
linear classifiers, and compositional neural networks. We find that
Tree-Structured Long Short-Term Memory networks (TreeLSTMs;
\citealt{tai2015improved,le2015compositional}) achieve the best
performance. In addition, we enhance the case for TreeLSTMs by showing
that its representations, trained on our corpus, perform well in a
disparate set of additional semantic tasks.  The success of these
transfer-learning experiments is also a testament to the centrality
of NLI in semantics.


%%%%%%%%%%%%%%%%%%%%%%%%%%%%%%%%%%%%%%%%%%%%%%%%%%%%%%%%%%%%%%%%%%%%%%


% One point of comparison that might be good to set up (maybe not in these words, this is Sam's sketch):

% Translation can also be used to train/evaluate NNs, and also demands some degree of sensitivity to compositional syntactic and semantic structure. Plus, it's easier to get good data for that task. But NLI is the better benchmark for developing NNs for language understanding, because:

% (i) Typical translation tasks require natural language generation, which is a separate difficult problem that must be learned in parallel with the semantic encoding task of interest, making results harder to interpret. We can just a vanilla well-understood classifier on top of our sentence model.

% (ii) Contradiction vs. entailment decisions in particular specifically target the abilities of NN models to learn lexical and phrasal representations (like alternation) that don't resemble similarity, either in their correlation with distributional information or their transitivity behavior. MT doesn't seem to have a good parallel to this. Since modeling similarity is almost the only aspect of NN behavior in NLP that's reasonably well understood and basically known to work, using a benchmark that explicitly demands something more sophisticated than this is likely to pay off by better exposing the weaknesses of current standard models.

% It might be also worth making an explicit comparison with sentiment as a benchmark, but that's low-hanging fruit.